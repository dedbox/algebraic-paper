\documentclass[letterpaper,12pt]{article}

\title{Algebraic structure for untyped Racket}
\author{Eric Griffis \\ dedbox@gmail.com}

% Page Layout

\usepackage[margin=1in]{geometry}

\usepackage{parskip}
\setlength{\parindent}{0pt}

% Content Layout

\usepackage{array}

\newenvironment{Grammar}
{
  \begin{tabular*}{\linewidth}{
    >{$}l<{$}
    >{$}c<{$}
    >{$}l<{$}
    @{\extracolsep{\fill}}
    r}
}{
  \end{tabular*}
}

\newcommand\OR{\ensuremath{~|~}}

% Notation

\usepackage{upgreek}
\usepackage{amsmath,amssymb}
\usepackage{semantic}
\usepackage{colonequals}

\mathlig{::=}{\coloncolonequals}
\mathlig{+->}{\mapsto}
\mathlig{>>}{\leadsto^\text{*}}
\mathlig{|-}{\vdash}
\mathlig{|=}{\vDash}
\mathlig{~>}{\leadsto}
\mathlig{<>}{\diamond}
\mathlig{**}{\times}
\mathlig{<}{\langle}
\mathlig{>}{\rangle}
\mathlig{=:}{\text{\texttt{=}}}
\mathlig{>:}{\text{\texttt{>}}}
\mathlig{<:}{\text{\texttt{<}}}
\mathlig{+:}{\text{\texttt{+}}}
\mathlig{-:}{\text{\texttt{-}}}
\mathlig{*:}{\text{\texttt{*}}}
\mathlig{/:}{\text{\texttt{/}}}
\mathlig{==}{\text{\texttt{==}}}
\mathlig{++}{\text{\texttt{++}}}
\mathlig{//}{\text{\texttt{//}}}

\newcommand\lt{\mathligsoff <}
\newcommand\gt{\mathligsoff >}

\def\Mac{\upmu}
\def\Fun{\upphi}
\def\Con{\delta}

% Inference Rules

\usepackage{mathpartir}

\newcommand\Rule[1]{\text{rule \RULE{#1}}}
\newcommand\RULE[1]{\text{\textsc{#1}}}
\newcommand\Infer[3][]{\inferrule{#2}{#3}~\RULE{#1}}

% Model

\newcommand\X[1]{\text{#1}}

% \newcommand\NullaryConstructor[1]{\def\#1{#1}}

\def\Succ{\X{Succ}~}
\def\Zero{\X{Zero}}
\def\True{\X{True}}
\def\False{\X{False}}
\def\Nil{\X{Nil}}
\def\Cons{\X{Cons}~}
\def\Loop{\X{Loop}~}

\def\add{\X{add}~}
\def\mul{\X{mul}~}
\def\Neg{\X{not}~}
\def\lan{\X{and}~}
\def\ior{\X{or}~}
\def\xor{\X{xor}~}
\def\lst{\X{list}~}
\def\rev{\X{rev}~}
\def\reverse{\X{reverse}~}
\def\append{\X{append}~}
\def\map{\X{map}~}
\def\Let{\X{let}~}
\def\Letrec{\X{letrec}~}
\def\fix{\X{fix}~}
\def\If{~\X{if}~}
\def\Fib{\X{fib}~}

\DeclareMathOperator{\match}{match}
\DeclareMathOperator{\vars}{vars}
\DeclareMathOperator{\hcall}{call}

% Large delimiters

\usepackage{mleftright}

\newenvironment{Clauses}[1][c|c]{
  \mleft[
    \hspace{-0.3em}
    \begin{array}{#1}
}{
    \end{array}
    \hspace{-0.3em}
  \mright]
}

% References

% \usepackage{natbib}

% ------------------------------------------------------------------------------

\begin{document}

\maketitle

\begin{abstract}
  A language for defining simple macros and pure functions with pattern-based
  syntax, along with first-class data constructors. Type system not included.
\end{abstract}

\section{The core model}
\label{sec:core-model}

\begin{minipage}[t]{0.45\linewidth}
  \vspace{0pt}
  \begin{Grammar}
    t
    & ::= & t~t      & application     \\
    & \OR & t;t      & sequence        \\
    & \OR & \Mac p.t & macro clause    \\
    & \OR & \Fun p.t & function clause \\
    & \OR & x        & variable        \\
    & \OR & \Con     & constructor     \\
    & \OR & <>       & null            \\
    v
    & ::= & \Mac p.t;\cdot & macro       \\
    & \OR & \Fun p.t;\cdot & function    \\
    & \OR & \Con           & constructor \\
    & \OR & \Con~(v;\cdot) & instance    \\
    & \OR & <>             & null        \\
  \end{Grammar}
\end{minipage}
\hfill
\begin{minipage}[t]{0.45\linewidth}
  \vspace{0pt}
  \begin{Grammar}
    p
    & ::= & p~p  & application \\
    & \OR & p;p  & sequence    \\
    & \OR & \_   & wildcard    \\
    & \OR & x    & variable    \\
    & \OR & \Con & constructor \\
    & \OR & <>   & null        \\
  \end{Grammar}
\end{minipage}

A \emph{constructor} ($\Con$) identifies a data type. Constructors are first
class; they can be matched against, collected into lists, and passed between
functions. A constructor applied to a value is an \emph{instance} of the
denoted type.

\begin{mathpar}
  \Infer[Seq1]{
    t_1 ~> t_1'
  }{
    t_1;t_2 ~> t_1';t_2
  }

  \Infer[Seq2]{
    t_2 ~> t_2'
  }{
    v_1;t_2 ~> v_1;t_2'
  }
\end{mathpar}

Sequences ($t;t$) are evaluated eagerly from left to right. Sequences collect
multiple clauses into a single macro or function. A sequence is \emph{uniform}
($t;\cdot$) when every sub-term has the same shape.

\begin{mathpar}
  \Infer[App1]{
    t_1 ~> t_1'
  }{
    t_1~t_2 ~> t_1'~t_2
  }

  \Infer[AppM]{
    p_{11} \times t_2 = \sigma \\
    \sigma(t_{12}) = t_{12}'
  }{
    (\Mac p_{11}.t_{12})~t_2 ~> t_{12}'
  }

  \Infer[App2]{
    v_1 \not\sim \Mac p.t;\cdot \\
    t_2 ~> t_2'
  }{
    v_1~t_2 ~> v_1~t_2'
  }

  \Infer[AppF]{
    p_{11} \times v_2 = \sigma \\
    \sigma(t_{12}) = t_{12}'
  }{
    (\Fun p_{11}.t_{12})~v_2 ~> t_{12}'
  }
\end{mathpar}

A \emph{macro clause} ($\Mac p.t$) is a lambda abstraction that can reject or
deconstruct a term by pattern matching against its unevaluated argument. A
\emph{macro} ($\Mac p.t;\cdot$) is a macro clause or a uniform sequence of
macro clauses. A macro gets stuck if every clause rejects the argument term.

A \emph{function clause} ($\Fun p.t$) is a lambda abstraction that can reject
or deconstruct a value by pattern matching against its fully-evaluated
argument. A \emph{function} ($\Fun p.t;\cdot$) is a function clause or a uniform
sequence of function clauses. A function gets stuck if every clause rejects
the argument value.

An \emph{application} ($t~t$) is evaluated quasi-eagerly, starting on the
left. If the left side reduces to a macro, it is a \emph{macro application}
and its patterns are matched against the un-evaluated right side. Otherwise,
the right side is evaluated; if the left side reduced to a function, it is a
\emph{function application} and its patterns are matched against the
evaluation result. In either case, if the match succeeds, any bound pattern
variables are substituted in the consequent term, which becomes the result of
the application.

\begin{mathpar}
  \Infer[Mac1]{
    (\Mac p_1.t_2)~t_4 ~> t_2'
  }{
    (\Mac p_1.t_2;v_3)~t_4 ~> t_2'
  }

  \Infer[Mac2]{
    p_1 ** t_4 = <>
  }{
    (\Mac p_1.t_2;v_3)~t_4 ~> v_3~t_4
  }

  \\

  \Infer[Fun1]{
    (\Fun p_1.t_2)~v_4 ~> t_2'
  }{
    (\Fun p_1.t_2;v_3)~v_4 ~> t_2'
  }

  \Infer[Fun2]{
    p_1 ** v_4 = <>
  }{
    (\Fun p_1.t_2;v_3)~v_4 ~> v_3~v_4
  }
\end{mathpar}

The clauses of a uniform sequence are applied in order, from left to right.
The first successful match determines the result.

\subsection{Pattern matching}
\label{sec:struct-patt-match}

\begin{mathpar}
  \Infer{
    p_1 ** t_1 = \sigma_1 \\
    p_2 ** t_2 = \sigma_2
  }{
    (p_1~p_2) ** (t_1~t_2) = \sigma_1 \cup \sigma_2
  }

  \Infer{
    p_1 ** t_1 = \sigma_1 \\
    p_2 ** t_2 = \sigma_2
  }{
    (p_1;p_2) ** (t_1;t_2) = \sigma_1 \cup \sigma_2
  }

  \Infer{}{x_1 ** t_2 = \{x_1 +-> t_2\}}

  \Infer{
    \Con_1 = \Con_2
  }{
    \Con_1 ** \Con_2 = \{\}
  }

  \Infer{}{\_ ** t_1 = \{\}}

  \Infer{}{<> ** <> = \{\}}
\end{mathpar}

A \emph{constructor pattern} ($\Con$) matches any constructor with the same
name. A \emph{null pattern} ($<>$) matches only the null term. A
\emph{wildcard} ($\_$) matches anything. A \emph{variable} ($x$) matches
anything and binds itself to the matched term. An \emph{application pattern}
($p~p$) or \emph{sequence pattern} ($p;p$) matches its sub-patterns against
the sub-terms of an application and passes along any variable bindings.

\subsection{Variable substitution}
\label{sec:vari-subst}

\begin{mathpar}
  \Infer{}{\sigma(t_1~t_2) = \sigma(t_1)~\sigma(t_2)}

  \Infer{}{\sigma(t_1;t_2) = \sigma(t_1);\sigma(t_2)}

  \Infer{
    \sigma' = \sigma \setminus \vars(p_1) \\
    t_2' = \sigma'(t_2)
  }{
    \sigma(\Mac p_1.t_2) = \Mac p_1.t_2'
  }

  \Infer{
    \sigma' = \sigma \setminus \vars(p_1) \\
    t_2' = \sigma'(t_2)
  }{
    \sigma(\Fun p_1.t_2) = \Fun p_1.t_2'
  }

  \Infer{
    (x_1 +-> t) \in \sigma
  }{
    \sigma(x_1) = t
  }

  \Infer{}{\sigma(\delta_1) = \delta_1}

  \Infer{}{\sigma(<>) = <>}
\end{mathpar}

\subsection{Examples}
\label{sec:core-examples}

\subsubsection{Numbers}
\label{sec:core-example-numbers}

\begin{align*}
  \X{add} & = \Fun(a~\Zero).a;\Fun(a~(\Succ b)).\Succ (\add (a~b))
\end{align*}

\begin{math}
  \add ((\Succ \Zero)~(\Succ (\Succ \Zero)))    \\
  ~> \Succ (\add ((\Succ \Zero)~(\Succ \Zero))) \\
  ~> \Succ (\Succ (\add ((\Succ \Zero)~\Zero))) \\
  ~> \Succ (\Succ (\Succ \Zero))
\end{math}

\begin{align*}
  \X{mul} & = \Fun(a~\Zero).\Zero;\Fun(a~(\Succ b)).\add (a~(\mul (a~b)))
\end{align*}

Denote by $N \in \mathbb{N}$ a series of $N$ $\Succ$\!s and a $\Zero$.

\begin{minipage}[t]{0.45\linewidth}
  \begin{math}
    \mul (2~3)                                     \\
    ~> \add (2~(\mul (2~2)))                       \\
    ~> \add (2~(\add (2~(\mul (2~1)))))            \\
    ~> \add (2~(\add (2~(\add (2~(\mul (2~0))))))) \\
    ~> \add (2~(\add (2~(\add (2~0)))))            \\
    ~> \add (2~(\add (2~2)))                       \\
    ~> \add (2~(\Succ (\add (2~1))))               \\
    ~> \add (2~(\Succ (\Succ (\add (2~0)))))       \\
    ~> \add (2~4)
  \end{math}
\end{minipage}
\hfill
\begin{minipage}[t]{0.45\linewidth}
  \begin{math}
    \\
    ~> \Succ (\add (2~3))                         \\
    ~> \Succ (\Succ (\add (2~2)))                 \\
    ~> \Succ (\Succ (\Succ (\add (2~1))))         \\
    ~> \Succ (\Succ (\Succ (\Succ (\add (2~0))))) \\
    ~> 6
  \end{math}
\end{minipage}

\subsubsection{Booleans}
\label{sec:core-example-booleans}

\begin{align*}
  \X{not} & = \Fun\False.\True;\Fun\_.\False           \\
  \X{and} & = \Mac(a~b).(\Fun\False.\False;\Fun\_.b)~a \\
  \X{or}  & = \Mac(a~b).(\Fun\False.b;\Fun x.x)~a      \\
  \X{xor} & = \Mac(a~b).(\Fun\False.b;\Fun x.\lan ((\Neg b)~x))~a
\end{align*}

\begin{math}
  \ior ((\Neg \True)~(\lan ((\xor (\True~\True))~\True))) \\
  ~> \ior (\False~(\lan ((\xor (\True~\True))~\True)))    \\
  ~> \lan ((\xor (\True~\True))~\True)                    \\
  ~> \lan ((\lan ((\Neg \True)~\True))~\True)             \\
  ~> \lan ((\lan (\False~\True))~\True)                   \\
  ~> \lan (\False~\True)                                  \\
  ~> \False
\end{math}

\subsubsection{Lists}
\label{sec:core-example-lists}

\begin{align*}
  \X{list} & = \Mac (x~<>).\Cons (x~\Nil);\Mac(x~xs).\Cons (x~(\lst xs))
\end{align*}

\begin{math}
  \lst (1~(2~(3~<>)))                    \\
  ~> \Cons (1~(\lst (2~(3~<>))))         \\
  ~> \Cons (1~(\Cons (2~(\lst (3~<>))))) \\
  ~> \Cons (1~(\Cons (2~(\Cons (3~\Nil)))))
\end{math}

\begin{align*}
  \X{rev} & = \Fun(\Nil~a).a;\Fun((\Cons (y~ys))~a).\rev (ys~(\Cons (y~a))) \\
  \X{reverse} & = \Fun xs.\rev (xs~\Nil)
\end{align*}

\begin{math}
  \reverse (\Cons (1~(\Cons (2~(\Cons (3~\Nil))))))       \\
  ~> \rev ((\Cons (1~(\Cons (2~(\Cons (3~\Nil))))))~\Nil) \\
  ~> \rev ((\Cons (2~(\Cons (3~\Nil))))~(\Cons (1~\Nil))) \\
  ~> \rev ((\Cons (3~\Nil))~(\Cons (2~(\Cons (1~\Nil))))) \\
  ~> \rev (\Nil~(\Cons (3~(\Cons (2~(\Cons (1~\Nil))))))) \\
  ~> \Cons (3~(\Cons (2~(\Cons (1~\Nil)))))
\end{math}

\begin{align*}
  \X{append} & = \Fun(\Nil~ys).ys;\Fun((\Cons (x~xs))~ys).\Cons (x~(\append (xs~ys)))
\end{align*}

\begin{math}
  \append ((\Cons (1~(\Cons (2~\Nil))))~(\Cons (3~(\Cons (4~\Nil)))))    \\
  ~> \Cons (1~(\append ((\Cons (2~\Nil))~(\Cons (3~(\Cons (4~\Nil))))))) \\
  ~> \Cons (1~(\Cons (2~(\append (\Nil~(\Cons (3~(\Cons (4~\Nil))))))))) \\
  ~> \Cons (1~(\Cons (2~(\Cons (3~(\Cons (4~\Nil)))))))
\end{math}

\begin{align*}
  \X{map} & = \Fun(\_~\Nil).\Nil;\Fun(f~(\Cons (x~xs))).\Cons ((f~x)~(\map (f~xs)))
\end{align*}

\begin{math}
  \map (\Succ (\Cons (3~(\Cons (2~(\Cons (1~\Nil)))))))    \\
  ~> \Cons (4~(\map (\Succ (\Cons (2~(\Cons (1~\Nil))))))) \\
  ~> \Cons (4~(\Cons (3~(\map (\Succ (\Cons (1~\Nil))))))) \\
  ~> \Cons (4~(\Cons (3~(\Cons (2~(\map (\Succ \Nil))))))) \\
  ~> \Cons (4~(\Cons (3~(\Cons (2~\Nil)))))
\end{math}

\section{The extended model}
\label{sec:ext-model}

\subsection{Extended syntax}
\label{sec:ext-syntax}

\begin{mathpar}
  t_1~t_2~\cdots~t_n ~> t_1~(t_2~\cdots~t_n)

  n \ge 2

  p_1~p_2~\cdots~p_n ~> p_1~(p_2~\cdots~p_n) \\
  
  t_1;t_2;\cdots;t_n ~> t_1;(t_2;\cdots;t_n)

  p_1;p_2;\cdots;p_n ~> p_1;(p_2;\cdots;p_n) \\

  \Mac \begin{Clauses}
    p_1 & t_1 \\
    p_2 & t_2 \\
    \vdots & \vdots \\
    p_n & t_n
  \end{Clauses}
  ~>
  \Mac p_1.t_1;\Mac \begin{Clauses}
    p_2 & t_2 \\
    \vdots & \vdots \\
    p_n & t_n
  \end{Clauses}

  \Mac \begin{Clauses} p_1 & t_1 \end{Clauses} ~> \Mac p_1.t_1

  \Fun \begin{Clauses}
    p_1 & t_1 \\
    p_2 & t_2 \\
    \vdots & \vdots \\
    p_n & t_n
  \end{Clauses}
  ~>
  \Fun p_1.t_1;\Fun \begin{Clauses}
    p_2 & t_2 \\
    \vdots & \vdots \\
    p_n & t_n
  \end{Clauses}

  \Fun \begin{Clauses} p_1 & t_1 \end{Clauses} ~> \Fun p_1.t_1
\end{mathpar}

\subsection{Let bindings}
\label{sec:ext-let-bindings}

\begin{mathpar}
  \X{fix} = \Fun f.(\Fun x.f~\Fun y.(x~x)~y)~(\Fun x.f~\Fun y.(x~x)~y)

  \X{let} \approx \Mac \begin{Clauses}[cc|c]
    (p~t)  & body & (\Fun p.body)~t \\
    (p~t;cs) & body & \Let (p~t)~\Let cs~body
  \end{Clauses}

  \X{letrec} \approx \Mac \begin{Clauses}[cc|c]
    (p~t)    & body & \Let (p~\fix \Fun p.body)~t \\
    (p~t;cs) & body & \Letrec (p~t)~\Letrec cs~body
  \end{Clauses}
\end{mathpar}

\subsection{Examples}
\label{sec:ext-examples}

\subsubsection{Numbers}
\label{sec:ext-example-numbers}

\begin{mathpar}
  \X{add} = \Fun \begin{Clauses}[cc|c]
    a & \Zero   & a \\
    a & \Succ b & \Succ \add a~b
  \end{Clauses}

  \X{mul} = \Fun \begin{Clauses}[cc|c]
    a & \Zero   & \Zero \\
    a & \Succ b & \add a~\mul a~b
  \end{Clauses}
\end{mathpar}

\begin{math}
  \add (\Succ \Zero)~\Succ \Succ \Zero    \\
  ~> \Succ \add (\Succ \Zero)~\Succ \Zero \\
  ~> \Succ \Succ \add (\Succ \Zero)~\Zero \\
  ~> \Succ \Succ \Succ \Zero
\end{math}

\begin{minipage}[t]{0.45\linewidth}
  \begin{math}
    \mul 2~3                         \\
    ~> \add 2~\mul 2~2               \\
    ~> \add 2~\add 2~\mul 2~1        \\
    ~> \add 2~\add 2~\add 2~\mul 2~0 \\
    ~> \add 2~\add 2~\add 2~0        \\
    ~> \add 2~\add 2~2               \\
    ~> \add 2~\Succ \add 2~1         \\
    ~> \add 2~\Succ \Succ \add 2~0
  \end{math}
\end{minipage}
\hfill
\begin{minipage}[t]{0.45\linewidth}
  \begin{math}
    \\
    ~> \add 2~4                         \\
    ~> \Succ \add 2~3                   \\
    ~> \Succ \Succ \add 2~2             \\
    ~> \Succ \Succ \Succ \add 2~1       \\
    ~> \Succ \Succ \Succ \Succ \add 2~0 \\
    ~> 6
  \end{math}
\end{minipage}

\subsubsection{Booleans}
\label{sec:ext-example-booleans}

\begin{mathpar}
  \X{not} = \Fun \begin{Clauses}
    \False & \True \\
    \_     & \False
  \end{Clauses}
  
  \X{and} = \Mac(a~b).\Fun \begin{Clauses}
    \False & \False \\
    \_     & b
  \end{Clauses}~a \\

  \X{or} = \Mac(a~b).\Fun \begin{Clauses}
    \False & b \\
    x      & x
  \end{Clauses}~a

  \X{xor} = \Mac(a~b).\Fun \begin{Clauses}
    \False & b \\
    x      & \lan (\Neg b)~x
  \end{Clauses}~a
\end{mathpar}

\begin{math}
  \ior (\Neg \True)~\lan (\xor \True~\True)~\True \\
  ~> \ior \False~\lan (\xor \True~\True)~\True    \\
  ~> \lan (\xor \True~\True)~\True                  \\
  ~> \lan (\lan (\Neg \True)~\True)~\True           \\
  ~> \lan (\lan \False~\True)~\True                 \\
  ~> \lan \False~\True                              \\
  ~> \False
\end{math}

\subsubsection{Lists}
\label{sec:ext-example-lists}

\begin{mathpar}
  \X{list} = \Mac \begin{Clauses}[cc|c]
    x & <> & \Cons (x;\Nil) \\
    x & xs & \Cons (x;\lst xs)
  \end{Clauses}
\end{mathpar}

\begin{math}
  \lst 1~2~3~<>                   \\
  ~> \Cons (1;\lst 2~3~<>)        \\
  ~> \Cons (1;\Cons (2;\lst 3))<> \\
  ~> \Cons (1;\Cons (2;\Cons (3;\Nil)))
\end{math}

\begin{mathpar}
  \X{reverse} = \Fun xs.\Letrec \mleft(rev~\Fun \begin{Clauses}[cc|c]
    \Nil           & a & a \\
    (\Cons (y;ys)) & a & rev~ys~\Cons (y;a)
  \end{Clauses}\mright)~rev~xs~\Nil
\end{mathpar}

\begin{math}
  \reverse \Cons (1;\Cons (2;\Cons (3;\Nil)))      \\
  ~> rev~(\Cons (1;\Cons (2;\Cons 3~\Nil)))~\Nil   \\
  ~> rev~(\Cons (2;\Cons (3;\Nil)))~\Cons (1;\Nil) \\
  ~> rev~(\Cons (3;\Nil))~\Cons (2;\Cons (1;\Nil)) \\
  ~> rev~\Nil~\Cons (3;\Cons (2;\Cons (1;\Nil)))   \\
  ~> \Cons (3;\Cons (2;\Cons (1;\Nil)))
\end{math}

\begin{mathpar}
  \X{append} = \Fun \begin{Clauses}[cc|c]
    \Nil           & ys & ys \\
    (\Cons (x;xs)) & ys & \Cons (x;\append xs~ys)
  \end{Clauses}
\end{mathpar}

\begin{math}
  \append (\Cons (1;\Cons (2;\Nil)))~\Cons (3;\Cons (4;\Nil))    \\
  ~> \Cons (1;\append (\Cons (2;\Nil))~\Cons (3;\Cons (4;\Nil))) \\
  ~> \Cons (1;\Cons (2;\append \Nil~\Cons (3;\Cons (4;\Nil))))   \\
  ~> \Cons (1;\Cons (2;\Cons (3;\Cons (4;\Nil))))
\end{math}

\begin{mathpar}
  \X{map} = \Fun \begin{Clauses}[cc|c]
    \_ & \Nil         & \Nil \\
    f  & \Cons (x;xs) & \Cons (f~x;\map f~xs)
  \end{Clauses}
\end{mathpar}

\begin{math}
  \map \Succ \Cons (3;\Cons (2;\Cons (1;\Nil)))    \\
  ~> \Cons (4;\map \Succ \Cons (2;\Cons (1;\Nil))) \\
  ~> \Cons (4;\Cons (3;\map \Succ \Cons (1;\Nil))) \\
  ~> \Cons (4;\Cons (3;\Cons (2;\map \Succ \Nil))) \\
  ~> \Cons (4;\Cons (3;\Cons (2;\Nil)))
\end{math}

\section{Host integration}
\label{sec:host-integration}

\subsection{Primitive data}
\label{sec:host-literals}

\begin{mathpar}
  t ::= \ldots \OR \ell

  p ::= \ldots \OR \ell

  v ::= \ldots \OR \ell

  \ell ::= <> \\

  \Infer{
    \ell_1 = \ell_2
  }{
    \ell_1 ** \ell_2 = \{\}
  }
\end{mathpar}

\subsubsection{Procedures}
\label{sec:host-procedures}

\begin{mathpar}
  t ::= \ldots \OR f

  p ::= \ldots \OR f

  v ::= \ldots \OR f~\ell

  f ::= \ldots \\

  \Infer[Pro]{
    \hcall(f_1,\ell_{21},\ell_{22},\ldots,\ell_{2n}) = \ell_1',\ell_2',\ldots,\ell_m' \\
    n,m \ge 1
  }{
    f_1~\ell_{21}~\ell_{22}~\cdots~\ell_{2n} ~> \ell_1'~\ell_2'~\cdots~\ell_m'
  }
\end{mathpar}

\subsubsection{Numbers}
\label{sec:host-numbers}

\begin{mathpar}
  \ell ::= \ldots \OR n

  f ::= \ldots \OR =: \OR >: \OR <: \OR +: \OR -: \OR *: \OR /:
\end{mathpar}

\begin{mathpar}
  \X{fib} = \Fun \begin{Clauses}
    n \If <:~n~2 & 1                            \\
    n            & +:~(\Fib -:~n~1)~\Fib -:~n~2 \\
  \end{Clauses}
\end{mathpar}

\begin{math}
  \Fib 3 \\
  ~> +:~(\Fib -:~3~1)~\Fib -:~3~2 \\
  ~> +:~(\Fib 2)~\Fib -:~3~2 \\
  ~> +:~(+:~(\Fib -:~2~1)~\Fib -:~2~2)~\Fib -:~3~2 \\
  ~> +:~(+:~(\Fib 1)~\Fib -:~2~2)~\Fib -:~3~2 \\
  ~> +:~(+:~1~\Fib -:2~2)~\Fib -:~3~2 \\
  ~> +:~(+:~1~\Fib 0)~\Fib -:~3~2 \\
  ~> +:~(+:~1~1)~\Fib -:~3~2 \\
  ~> +:~2~\Fib -:~3~2 \\
  ~> +:~2~\Fib 1 \\
  ~> +:~2~1 \\
  ~> 3
\end{math}

% \subsubsection{Characters}
% \label{sec:host-characters}

% \begin{mathpar}
%   \ell ::= \ldots \OR c
% \end{mathpar}

% \subsubsection{Regular expressions}
% \label{sec:regular-expressions}

% \begin{mathpar}
%   \ell ::= \ldots \OR q

%   \Infer{
%     \match_\X{rx}(q_1,t_2) = \sigma
%   }{
%     q_1 ** t_2 = \sigma
%   }
% \end{mathpar}

\subsubsection{Booleans}
\label{sec:host-booleans}

\begin{mathpar}
  \ell ::= \ldots \OR \top \OR \bot
\end{mathpar}

% (letrec ([xor μ (a b) (fun [#f b] [x and (not b) x]) a])
%   or (not #t) and (xor #t #t) #t)

  % \X{not} = \Fun \begin{Clauses}
  %   \False & \True \\
  %   \_     & \False
  % \end{Clauses}
  
  % \X{and} = \Mac(a~b).\Fun \begin{Clauses}
  %   \False & \False \\
  %   \_     & b
  % \end{Clauses}~a \\

  % \X{or} = \Mac(a~b).\Fun \begin{Clauses}
  %   \False & b \\
  %   x      & x
  % \end{Clauses}~a

  % \X{xor} = \Mac(a~b).\Fun \begin{Clauses}
  %   \False & b \\
  %   x      & \lan (\Neg b)~x
  % \end{Clauses}~a

\begin{mathpar}
  \X{not} = \Fun \begin{Clauses}
    \bot & \top \\
    \_   & \bot
  \end{Clauses}

  \X{and} = \Mac(a~b).\Fun \begin{Clauses}
    \bot & \bot \\
    \_   & b
  \end{Clauses}~a

  \X{or} = \Mac(a~b).\Fun \begin{Clauses}
    \bot & b \\
    x    & x
  \end{Clauses}~a

  \X{xor} = \Mac(a~b).\Fun \begin{Clauses}
    \bot & b \\
    x    & \lan (\Neg b)~x
  \end{Clauses}~a
\end{mathpar}

\begin{math}
  \ior (\Neg \top)~\lan (\xor \top~\top)~\top \\
  ~> \ior \bot~\lan (\xor \top~\top)~\top \\
  ~> \lan (\xor \top~\top)~\top \\
  ~> \lan (\lan (\Neg \top)~\top)~\top \\
  ~> \lan (\lan \bot~\top)~\top \\
  ~> \lan \bot~\top \\
  ~> \bot
\end{math}

\subsubsection{Lists}
\label{sec:host-lists}

\begin{mathpar}
  \ell ::= \ldots \OR \ell~\ell
\end{mathpar}

% \subsubsection{Strings}
% \label{sec:host-strings}

% \begin{mathpar}
%   \ell ::= \ldots \OR s

%   f ::= \ldots \OR == \OR ++ \OR //
% \end{mathpar}

% \subsection{Lists}
% \label{sec:host-lists}

% \begin{mathpar}
%   t ::= \ldots \OR t :: t

%   p ::= \ldots \OR p :: p

%   \ell ::= \ldots \OR \ell :: \ell \\

%   \Infer{
%     t_1 ~> t_1'
%   }{
%     t_1 :: t_2 ~> t_1' :: t_2
%   }

%   \Infer{
%     t_2 ~> t_2'
%   }{
%     v_1 :: t_2 ~> v_1 :: t_2'
%   } \\

%   \Infer{
%     p_1 ** t_1 = \sigma_1 \\
%     p_2 ** t_2 = \sigma_2
%   }{
%     (p_1 :: p_2) ** (t_1 :: t_2) = \sigma_1 \cup \sigma_2
%   }
% \end{mathpar}

% \begin{mathpar}
%   \X{lis} = \Mac
%   \begin{Clauses}[l|c]
%     t_1~<>  & t_1::<> \\
%     t_1~t_2 & t_1::\X{lis}~t_2
%   \end{Clauses}

%   \X{list} = \Fun
%   \begin{Clauses}[l|c]
%     v_1~<>  & v_1::<> \\
%     v_1~v_2 & v_1::\X{list}~v_2
%   \end{Clauses} \\

%   \X{ap} = \Mac \begin{Clauses}[l|c]
%     t_1::<>  & t_1 \\
%     t_1::t_2 & t_1~\X{ap}~t_2
%   \end{Clauses}

%   \X{app} = \Fun \begin{Clauses}[l|c]
%     v_1::<>  & v_1 \\
%     v_1::v_2 & v_1~\X{app}~v_2
%   \end{Clauses}
% \end{mathpar}

% \subsection{Arrays}
% \label{sec:host-arrays}

% \begin{mathpar}
%   t ::= \ldots \OR <t;\cdot>

%   p ::= \ldots \OR <p;\cdot>

%   v ::= \ldots \OR <v;\cdot>

%   \ell ::= \ldots \OR <\ell;\cdot> \\

%   \Infer[Vec]{
%     t_1 ~> t_1'
%   }{
%     <t_1> ~> <t_1'>
%   }

%   \Infer[VRef]{
%     0 \le n_2 \le p
%   }{
%     <v_0;v_1;\cdots;v_p>~n_2 ~> v_{n_2}
%   } \\

%   \Infer{
%     p_1 ** t_1 = \sigma_1 \\
%     <p_2> ** <t_2> = \sigma_2
%   }{
%     <p_1;p_2> ** <t_1;t_2> = \sigma_1 \cup \sigma_2
%   }

%   \Infer{
%     p_1 ** t_1 = \sigma
%   }{
%     <p_1> ** <t_1> = \sigma
%   }
% \end{mathpar}

% \subsection{Dictionaries}
% \label{sec:host-dictionaries}

% \begin{mathpar}
%   t ::= \ldots \OR x +-> t \OR [t]
  
%   p ::= \ldots \OR x +-> p \OR [p]
  
%   \ell ::= \ldots \OR [x +-> \ell;\cdot] \\

%   \Infer{
%     t_2 ~> t_2'
%   }{
%     x_1 +-> t_2 ~> x_1 +-> t_2'
%   }

%   \Infer{
%     t_1 ~> t_1'
%   }{
%     [t_1] ~> [t_1']
%   } \\

%   \Infer{
%     p_{12} ** t_{22} = \sigma
%   }{
%     (x_{11} +-> p_{12}) ** (x_{21} +-> t_{22}) = \{x_{11} +-> x_{21}\} \cup \sigma
%   }

%   \Infer{
%     p_1 ** t_1 = \sigma_1 \\
%     [p_2] ** [t_2] = \sigma_2
%   }{
%     [p_1;p_2] ** [t_1;t_2] = \sigma_1 \cup \sigma_2
%   }

%   \Infer{
%     p_1 ** t_1 = \sigma
%   }{
%     [p_1] ** [t_1] = \sigma
%   }
% \end{mathpar}

\subsection{Pattern Guards}
\label{sec:host-pattern-guards}

\begin{mathpar}
  p ::= \ldots \OR p \If t \\

  \Infer{
    p_1 ** t_3 = \sigma_1 \\
    \sigma_1(t_2) >> v_2 \ne \bot
  }{
    (p_1 \If t_2) ** t_3 = \sigma_1
  }
\end{mathpar}

% ------------------------------------------------------------------------------

% \bibliographystyle{alpha}
\bibliography{references.bib} 

\end{document}
